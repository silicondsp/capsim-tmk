
% Copyright (C)  2007-2018 Sasan Ardalan
%    Permission is granted to copy, distribute and/or modify this document
%    under the terms of the GNU Free Documentation License, Version 1.3
%    or any later version published by the Free Software Foundation;
%    with no Invariant Sections, no Front-Cover Texts, and no Back-Cover Texts.
%    A copy of the license is included in the section entitled "GNU
%    Free Documentation License".
 






%
% This document has designed to be read through as a PDF first.
% Click on the ``Typeset'' button in the toolbar to generate the PDF
% if it is not already visible.
%

%%% PREAMBLE %%%
% You probably want to skip to \begin{document} if this is your first time.

\documentclass[article,oneside]{memoir}
% This command goes at the beginning of every document
% [oneside,article] are two of many options that can be chosen
%   oneside makes each page have the same layout, for printing on only one side
%     of the paper (change it to twoside to see the difference)
%   article means we're writing a short document only and won't be using special
%     chapter headings
%   a4paper changes the page dimensions for A4 sized paper

%%% PACKAGES %%%

\usepackage{graphicx} % Add graphics capabilities
\usepackage{booktabs} % ``Proper'' table layout
\usepackage{amsmath}  % Better maths support
\usepackage[colorlinks=true,linkcolor=red]{hyperref} % Hyperlink capabilities

\usepackage{memhfixc} % This package is required to resolve incompatibilities
                      % with the memoir class & the hyperref package

%\usepackage{pdfsync}
%-----------   \usepackage{/Library/texmf/tex/latex/pdfsync}
% This package is used to tell TeXShop where things are in the PDF file.
% Command-click at any spot in the PDF and it will jump to the corresponding
% location in the source file.

%%% COMMANDS %%%




% Define the title, author and date of the document.
% If the date is undefined, the current date is substituted.
\title {PLL Design}
\author{Sasan Ardalan}
\date{ January 28, 2018}


%%% BODY OF THE DOCUMENT: %%%
\begin{document}

\maketitle

%\begin{figure}[hbtp]
%  \centering
%    \includegraphics[width=2in]{circuit}
%  \caption{Circuit }
%  \label{fig:circuit}
%\end{figure}


\begin{equation}
\tau= R_1C_1
\end{equation}

\begin{equation}
\omega_n= \frac{K}{2\zeta}
\end{equation}

\begin{equation}
\omega_n= \sqrt{ \frac{K_0I_p}{2\pi N C_1} }
\end{equation}


\begin{equation}
K=\frac{K_0I_pR_1}{2\pi N}
\end{equation}


\begin{equation}
\zeta= \frac{\sqrt{K\tau}}{2}
\end{equation}

\begin{equation}
\zeta= \frac{\sqrt{\frac{K_0I_pR_1}{2\pi N}R_1C_1}}{2}
\end{equation}

\begin{equation}
\zeta= \frac{1}{2}R_1\sqrt{\frac{K_0I_pC_1}{2\pi N}}
\end{equation}






\begin{equation}
Phase~Margin= \arctan{\left ( \sqrt{\frac{C_1}{C_2}+1}\right )}-\arctan{\left (\frac{1}{\sqrt{\frac{C_1}{C_2}+1}}\right )}
\end{equation}

\begin{equation}
BW= 2\omega_n\zeta
\end{equation}

\begin{equation}
K_{stable}=\frac{1}{\frac{\pi}{2\pi f_i\tau}\left ( 1+\frac{\pi}{2\pi f_i \tau} \right )}
\end{equation}

\begin{equation}
K= 2\omega_n\zeta
\end{equation}

\begin{equation}
K \tau= 4 \zeta^2
\end{equation}

\begin{equation}
\frac{K }{\tau}= \omega_n^2
\end{equation}



$\int$

\begin{equation}
T_w \propto  \frac{1}{C}
\end{equation}


\begin{equation}
 -\frac{1}{R_{SC}C_F} \int v(t)dt
\end{equation}


$v(t)$
\begin{equation}
K_d=\frac{I_p}{2\pi}
\end{equation}

\begin{equation}
Z_F(s)
\end{equation}


\begin{equation}
N(s)
\end{equation}


\begin{equation}
\Theta_o(s)
\end{equation}


\begin{equation}
\Theta_n(s)
\end{equation}

\begin{equation}
\Theta_i(s)
\end{equation}


\begin{equation}
\frac{K_0}{N}\frac{1}{s}
\end{equation}


\begin{equation}
\sqrt{E[\Theta^2_{tot}(nT)]}=\sqrt{\frac{1}{2K_L T}}\frac{2\pi \Delta \tau_{rms}}{T}
\end{equation}

\begin{equation}
K_L=\frac{K_0I_pR_1}{2\pi N}
\end{equation}

\begin{equation}
D(s) = s^2+s\zeta \omega_n + \omega_n^2
\end{equation}

\begin{equation}
\frac{1}{s^2+s\zeta \omega_n + \omega_n^2}
\end{equation}

\begin{equation}
e^{-\zeta \omega_n t} \sin ( \sqrt{1-\zeta^2}\omega_n t ) \frac{1}{\sqrt{1-\zeta^2} \omega_n }
\end{equation}



PAD Resistance
\begin{equation}
\frac{V_k(s)}{I_P(s)} =\frac{1}{(C_1+C_2)s}\frac{(1+sRC_1)(1+sR_PC_2)}{1+s\frac{C_1C_2(R_P+R)}{C_1+C_2}}
\end{equation}

\begin{equation}
VC_1
\end{equation}



\end{document}



